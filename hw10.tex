\documentclass[12pt]{article}
\usepackage{fullpage,hyperref}\setlength{\parskip}{3mm}\setlength{\parindent}{0mm}
\begin{document}

\begin{center}\bf
Homework 10 Solution - Yidan Xu

Linux and the open source software movement.

\end{center}

Linux is the dominant environment for scientific computing. For example, since 2017 all the 500 fastest  supercomputers have run some variant of Linux (\url{https://en.wikipedia.org/wiki/Linux}). As another example, most cloud servers are built on Linux, and Linux is therefore dominant for data science applications that involve cloud computing. At University of Michigan, the main resource for high performance computing is the Great Lakes Linux cluster. It should be apparent that Linux skills are useful for a research statistician, as soon as your data analysis or simulation study is too large for a laptop.

Linux expertise in this class ranges from novice to expert. Our goal is to advance our understanding and share knowledge.

Write brief answers to the following questions, by editing the tex file available at \url{https://github.com/ionides/810f21}, and submit the resulting pdf file via Canvas.

\begin{enumerate}

\item Linux, R and Python are all open source and free. 

(a) How do you think these projects led to high quality products given that the  usual financial incentives for building, coordinating and running a development team are missing?

Contribution by worldwide developers, for whom the developing of the software itself would be beneficial to their daily use.

(b) If developers are interested in making money, can they do this by writing free software? If so, how? If not, why do they do it?

As an example, dual licensing -- e.g. the company would offer a free version for individual user and paid version for the companies.
  
\item To what extent do you agree or disagree with the opinions at

\url{https://hub.packtpub.com/data-science-windows-big-no/}? 

I have been a Mac user and have heard that it is rather difficult to run through command lines using Windows. But I think some of the IDEs have made it more easier e.g. VScode.

\item If you are new to Linux, or if your Linux skills are limited to a handful of commands, read the introduction to command line Linux at
  
\url{https://tutorials.ubuntu.com/tutorial/command-line-for-beginners}

If you have a Mac, try out the commands on a Terminal app, which runs a version of Unix that works identically to Linux for many everyday purposes. If you run Windows, try out some commands on the Windows subsystem for Linux

\url{https://docs.microsoft.com/en-us/windows/wsl/}

You can also log in to the UM Linux login service, using SSH Secure Shell. For example from a Mac terminal, type

\texttt{ssh your\_uniqname@login.itd.umich.edu}

Optionally, if you find it an effective way to practice, play the terminus adventure at

\url{https://web.mit.edu/mprat/Public/web/Terminus/Web/main.html}

I was unaware of the meaning of some of the commands before and was directly copy-pasting from tutorials/stackoverflow whenever I need to install packages etc. In particular, caution should be practiced when using `sudo` or `su` in granting superuser privilege. 


\item If you are a relatively experienced Linux user, share some words of advice for beginners. How and why did you get started with Linux?  

NA

Optionally, you can also use this opportunity to learn some more Linux-related skills, e.g., from the Linux intermediate tutorials at

\url{https://www.linux.org/forums/linux-intermediate-tutorials.124/}


\end{enumerate}
\end{document}
